% Distributing
\section{Distributing}
\subsection{Register the application}
\begin{frame}[fragile]{Register the application}
    \begin{block}{PyPI account}
        Create a PyPI\footnote[1]{\href{https://pypi.python.org/pypi}{https://pypi.python.org/pypi}} account.
        Configure \texttt{.pypirc} file with PyPI credentials.
    \end{block}
    \pause
    \begin{block}{Application register}
        Use twine\footnote[2]{\href{https://github.com/pypa/twine}{https://github.com/pypa/twine}} with a package of your application to register it in PyPI:
        \begin{minted}{bash}
twine register dist/project-name.whl
        \end{minted}
    \end{block}
    \note{
        Hay que crear una cuenta en PyPI y definir los credenciales en el fichero de configuración .pypirc

        Para subir el paquete al repositorio se usa twine, principalmente por seguridad, ya que encripta las conexiones. Twine te permite registrar tu aplicación en el repositorio.
    }
\end{frame}

\subsection{Upload a new version}
\begin{frame}[fragile]{Upload a new version}
    \begin{block}{Upload packages}
        Use twine again to upload all packages to PyPI:
        \begin{minted}{bash}
twine upload dist/project-name.whl
twine upload dist/project-name.tar.gz
        \end{minted}
    \end{block}
    \note{
        Generalmente es una buena idea subir tanto un fichero wheel como un tar.gz, para intérpretes no compatibles.
    }
\end{frame}
 
