\documentclass[final, 9pt, svgnames]{beamerPerdy}
\usetheme[
    titleformat title=smallcaps,
    progressbar=frametitle,
    numbering=fraction
]{metropolis}

\usetikzlibrary{positioning,fit,arrows.meta}

%\setbeameroption{show notes on second screen=right}

\setdefaultlanguage{english}

% Comando para usar comentarios
\newcommand{\comment}[1]{\textcolor{comment}{\footnotesize{#1}\normalsize}}

\newcommand\ImageNode[3][]{
    \node[draw=mLightBrown,line width=1pt,#1] (#2) {\includegraphics[width=1cm,height=1cm]{#3}};
}


\title{Packaging and Distributing}
\subtitle{Create, package and distribute your own python application}
\author[J. A. Perdiguero López]{
José Antonio Perdiguero López\\
\href{http://www.perdy.top}{\scriptsize{\faGlobe\; http://www.perdy.top}}\\
\href{https://github.com/PeRDy}{\scriptsize{\faGithub\; https://github.com/PeRDy}}\\
\href{https://www.linkedin.com/in/perdy}{\scriptsize{\faLinkedin\; https://www.linkedin.com/in/perdy}}\\
\scriptsize{\faAt\; perdy.hh@gmail.com}}
\date{September 24, 2017}
\institute[Piksel]{Lead Developer @ Piksel, Machine Learning Team}

\begin{document}

% Title
\begin{frame}[noframenumbering, plain]
    \titlepage
\end{frame}

% Index
\begin{frame}[noframenumbering, plain]{Index}
    \begin{columns}[t]
        \begin{column}{5cm}
            \setbeamertemplate{section in toc}[circle]
            \tableofcontents[sections={1-2}]
        \end{column}
        \begin{column}{5cm}
            \setbeamertemplate{section in toc}[circle]
            \tableofcontents[sections={3-5}]
        \end{column}
    \end{columns}
\end{frame}

% Introduction 
\section{Introduction}
\subsection{Relational Databases}
\begin{frame}{Relational Databases}
    Relational databases are those based on \emph{Relational Model}, invented by \emph{E.F. Codd}.
    
    The relational model define that all the data is represented in terms of \alert{tuples}, grouped in \alert{relations}. These relations consists of a set of attributes and a set of $n$ tuples, and are called \alert{tables}.
\end{frame}
\begin{frame}{Relational Databases: Pros vs Cons}
    \begin{columns}[t]
        \begin{column}{.5\textwidth}
            \begin{exampleblock}{\huge\faThumbsOUp}
                \begin{itemize}
                    \item Structured data
                    \item Strong typing
                    \item Integrity and constraints
                    \item Indexes
                    \item Support
                    \item Fast querying
                    \item Maturity
                \end{itemize}
            \end{exampleblock}
        \end{column}
        \begin{column}{.5\textwidth}
            \begin{alertblock}{\huge\faThumbsODown}
                \begin{itemize}
                    \item Admin
                    \item Poor horizontal scaling
                    \item Schema and data migrations
                \end{itemize}
            \end{alertblock}
        \end{column}
    \end{columns}
\end{frame}

\subsection{NoSQL Databases}
\begin{frame}{NoSQL Databases}
    \begin{block}{Key-Value}
        Key-Value stores are the simplest NoSQL data stores. The value is a blob, and due to primary-key access they have a great performance and scalability.
    \end{block}
    \begin{block}{Document}
        Documents are the main concept in document databases. The database stores and retrieves documents, which can be XML, JSON, BSON, and so on.
    \end{block}
    \begin{block}{Column family}
        Column-family databases store data in column families as rows that have many columns associated with a row key.
    \end{block}
    \begin{block}{Graph}
        Graph databases allow you to store entities and relationships between these entities. Entities are also known as nodes, which have properties. Relations are known as edges that can have properties. 
    \end{block}
    
    \note {
        Memcached and Redis as \textbf{Key-Value} databases.
        
        MongoDB as \textbf{document} database.

        Cassandra is a \textbf{column family} database. These databases are like relational tables but each row can have a different subset of columns.
        
    In \textbf{graph} databases think of a node as an instance of an object in the application. OrientDB and Neo4j.
    }
\end{frame}
\begin{frame}{NoSQL Databases: Pros vs Cons}
    \begin{columns}[t]
        \begin{column}{.5\textwidth}
            \begin{exampleblock}{\huge\faThumbsOUp}
                \begin{itemize}
                    \item Fast writing
                    \item Horizontal scaling
                    \item Reduced admin tasks
                    \item Flexible
                    \item Mostly open-source
                \end{itemize}
            \end{exampleblock}
        \end{column}
        \begin{column}{.5\textwidth}
            \begin{alertblock}{\huge\faThumbsODown}
                \begin{itemize}
                    \item Slow querying
                    \item Support
                    \item Maturity
                \end{itemize}
            \end{alertblock}
        \end{column}
    \end{columns}
\end{frame}

\subsection{Relational vs NoSQL}
\begin{frame}[standout]
    \huge{Should I use a Relational or NoSQL database?}

    \huge{\faDatabase}
\end{frame}

% Creation
\section{Creation}
\subsection{Storing the project}
\begin{frame}{Storing the project}
    \begin{figure}
        \includegraphics[width=\textwidth,height=0.8\textheight,keepaspectratio]{create_repository.png}
    \end{figure}
    \note{
        Esta charla está completamente orientada a open source así que, ¿por qué no trabajar con GitHub?
    }
\end{frame}

\subsection{Create the project skeleton}
\begin{frame}[fragile]{Create the project skeleton}
    \begin{block}{Cookiecutter context}
        Define all variables needed by cookiecutter to properly create the project skeleton, these variables can be found in \textit{cookiecutter.json} file.
    \end{block}
    \pause
    \begin{block}{Create skeleton}
        Execute cookiecutter with previously defined context to create the project skeleton.
    \end{block}
    \pause
    \begin{block}{Commit \& push}
        Time to do your first commit and push to repository:
        \begin{minted}{bash}
git remote add origin git@github.com:PeRDy/foo.git
git commit -a -m "Initial commit"
git push
        \end{minted}
    \end{block}
    \note {
        Para crear el proyecto:
        \begin{enumerate}
            \item Definir el contexto necesario para Cookiecutter: nombre, autor, repositorio...
            \item Crear el esqueleto del proyecto con la plantilla.
            \item Commit iniciar y pushear.
        \end{enumerate}
    }
\end{frame}

\subsection{Project hierarchy}
\begin{frame}{Project hierarchy}
    \begin{columns}
        \begin{column}{0.2\textwidth}
            \begin{figure}
                \includegraphics[width=\textwidth,height=\textheight,keepaspectratio]{project_skeleton.png}
            \end{figure}
        \end{column}
        \begin{column}{0.80\textwidth}
            \begin{block}{Documentation folder}
                The place that keeps all the documentation source files as well as the doc config file.
            \end{block}
            \pause
            \begin{block}{Tests folder}
                All tests files are stored in a tests folder where tests collectors can gather them without problems.
            \end{block}
            \pause
            \begin{block}{Application folder}
                The application itself, the \emph{python package} distributed, and the same that other users will import in their applications.
            \end{block}
            \pause
            \begin{block}{Root files}
                Files that keeps in the root directory are usually:
                \begin{itemize}
                    \item Tools configuration.
                    \item Services configuration.
                    \item Build scripts.
                    \item Metadata.
                \end{itemize}
            \end{block}
        \end{column}
    \end{columns}
    \note{
        \begin{description}
            \item[Configuración de herramientas] Prospector, Pre-commit, Git.
            \item[Configuración de servicios] Travis.
            \item[Scripts de construcción] \texttt{build.py}, \texttt{setup.py}, \texttt{tox.ini}.
            \item[Metadatos] Readme, manifest, contributors, changelog, requirements, \texttt{setup.cfg}.
        \end{description}
    }
\end{frame}

\begin{frame}[fragile]{Relevant files I}
    \begin{block}{Manifest}
        This file, \texttt{MANIFEST.in}, with own syntax\footnote[1]{\href{https://docs.python.org/3/distutils/commandref.html#sdist-cmd}{https://docs.python.org/3/distutils/commandref.html#sdist-cmd}} defines the directories and files that will be included in the distributable package.
    \end{block}
    \pause
    \begin{block}{Requirements}
        List all requirements of your project, that are added as dependencies when installed. Usually requirements are splitted in two files: \texttt{requirements.txt} for real dependencies and \texttt{requirements-tests.txt} for dependencies necessaries to test the project.
    \end{block}
    \pause
    \begin{block}{Metadata}
        Metadata files: \texttt{README.rst}, \texttt{CONTRIBUTORS.rst}, \texttt{CHANGELOG.rst} and \texttt{LICENSE}.
    \end{block}
    \note{
        El fichero README puede ser escrito en reStructuredText o en Markdown. Este fichero, renderizado, será la página principal del proyecto en GitHub,
    }
\end{frame}

\begin{frame}{Relevant files II}
    \begin{block}{Tools and Services config}
        Configuration files for tools and services: \texttt{setup.cfg}, \texttt{.pre-commit-config.yaml}, \texttt{.prospector.yaml}, \texttt{.travis.yml}, \texttt{.gitignore}.
    \end{block}
    \pause
    \begin{block}{Setup}
        Main file that defines how the project will be packaged, gather metadata from other files and provides an interface to create distributable packages.
    \end{block}
    \pause
    \begin{block}{Tox}
        Tox file, \texttt{tox.ini}, defines the environments and commands that tox executes. In this case, defines an environment for each python version that should be tested, another for run lint tools and the last one for compile documentation.
    \end{block}
    \note{
        El fichero principal de configuración es \texttt{setup.cfg}, que es un fichero de estilo \texttt{.ini}, dividido en secciones con configuraciones para las diferentes herramientas como por ejemplo \emph{bumpversion}, \emph{pytest} y \emph{coverage}.

        El fichero \texttt{setup.py} contiene la versión actual del proyecto, su nombre y su lista de requirements.

        Tox está integrado con Travis.
    }
\end{frame}
\begin{frame}{Relevant files III}
    \begin{block}{Build}
        The build file, \texttt{build.py}, is the entrypoint for everything related to build, including \emph{testing}, \emph{packaging} and \emph{distributing}. This is a command line application using \emph{Clinner} that provides a set of utility commands such as:
        \begin{itemize}
            \item Run tests and code coverage.
            \item Run lint.
            \item Run tox.
            \item Create documentation.
            \item Upgrade version, create package and upload to pypi.
        \end{itemize}
    \end{block}
    \note {
        Estas son las funciones del script de Clinner, más adelante se muestra la salida de la ayuda del script.
    }
\end{frame}

% Packaging
\section{Packaging}
\subsection{Test your application}
\begin{frame}[fragile]{Test your application}
    \begin{block}{Development}
        Run tests while developing using nose.
        \begin{minted}{bash}
python build.py test
        \end{minted}
    \end{block}
    \pause
    \begin{block}{Multi-environment}
        Once development is done, run tests against all different interpreters supported to check compatibility.
        \begin{minted}{bash}
python build.py tox
        \end{minted}
    \end{block}
    \pause
    \begin{block}{Continuous Integration}
        Development is done, tests pass in every environment, so code can be uploaded to repository safely. Once a commit is done:
        \begin{description}
            \item[Travis] run tests in every environment and will notify in case any test didn't pass. When all tests pass,
            \item[Coveralls] records current code coverage. In the same commit,
            \item[ReadTheDocs] gets the code, build docs and updates the project's doc page.
        \end{description}
    \end{block}
\end{frame}

\subsection{Creating a package}
\begin{frame}[fragile]{Package types}
    \begin{block}{Egg}
        Source distribution.
        \begin{minted}{bash}
python setup.py sdist
        \end{minted}
    \end{block}
    \pause
    \begin{block}{Wheel}
        Built and binary distribution.
        \begin{minted}{bash}
python setup.py bdist_wheel
        \end{minted}
    \end{block}
    \note{
        There is two types of python packages: \emph{egg} and \emph{wheel}. Source distribution needs a build step, built or binary distributions only need to move the package in the right path. Case of \emph{numpy}, \emph{scipy} and \emph{pandas}.
    }
\end{frame}

\begin{frame}{Builder}
    \begin{figure}
        \includegraphics[width=\textwidth,height=0.8\textheight,keepaspectratio]{build_help.png}
    \end{figure}
\end{frame}


% Distributing
\section{Distributing}
\subsection{Register the application}
\begin{frame}[fragile]{Register the application}
    \begin{block}{PyPI account}
        Create a PyPI\footnote[1]{\href{https://pypi.python.org/pypi}{https://pypi.python.org/pypi}} account.
        Configure \texttt{.pypirc} file with PyPI credentials.
    \end{block}
    \pause
    \begin{block}{Application register}
        Use twine\footnote[2]{\href{https://github.com/pypa/twine}{https://github.com/pypa/twine}} with a package of your application to register it in PyPI:
        \begin{minted}{bash}
twine register dist/project-name.whl
        \end{minted}
    \end{block}
\end{frame}

\subsection{Upload a new version}
\begin{frame}[fragile]{Upload a new version}
    \begin{block}{Upload packages}
        Use twine again to upload all packages to PyPI:
        \begin{minted}{bash}
twine upload dist/project-name.whl
twine upload dist/project-name.tar.gz
        \end{minted}
    \end{block}
\end{frame}
 

% Conclusion
\section{Conclusion}
\subsection{Full workflow}
\begin{frame}[fragile]{Full workflow}
    \makebox[\textwidth][c]{
    \begin{tikzpicture}[
            node distance=0.8cm,
            >={Triangle[angle=60:1pt 2]},
            shorten >= 2pt,
            shorten <= 2pt,
            arrow/.style={
                ->,
                mLightBrown,
                line width=3pt
            }
        ]
        \ImageNode[label={Create Project}]{A}{python_logo.png}
        \ImageNode[label={Configure},right=of A]{B}{python_logo.png}
        \ImageNode[label={0:Code},right=of B]{C}{python_logo.png}
        \ImageNode[label={Upgrade Version},above right=of C]{D}{python_logo.png}
        \ImageNode[label={Package},right=of D]{E}{python_logo.png}
        \ImageNode[label={PyPI},right=of E]{F}{python_logo.png}

        \ImageNode[label={-90:GitHub},below right=of C]{G}{github_logo.png}
        \ImageNode[label={-90:Travis},right=of G]{H}{travis_logo.png}
        \ImageNode[label={-90:Codecov},right=of H]{I}{codecov_logo.png}
        \ImageNode[label={0:ReadTheDocs},below=of H]{J}{readthedocs_logo.png}

        \draw[arrow] (A) -- (B);
        \draw[arrow] (B) -- (C);
        \draw[arrow] (C) -- (D);
        \draw[arrow] (D) -- (E);
        \draw[arrow] (E) -- (F);
        \draw[arrow] (C) -- (G);
        \draw[arrow] (G) -- (H);
        \draw[arrow] (H) -- (I);
        \draw[arrow] (G) -- (J);

        \onslide<2->{
            \node (X) [draw=mLightGreen, fit= (A) (B), inner sep=0.55cm, thick, fill=mLightGreen, fill opacity=0.1] {};
            \node [yshift=1.5ex, mLightGreen] at (X.south) {Creation};
        }
        \onslide<3->{
            \node (Y) [draw=mLightGreen, fit= (D) (E) (F), inner sep=0.55cm, thick, fill=mLightGreen, fill opacity=0.1] {};
            \node [yshift=1.5ex, mLightGreen] at (Y.south) {Packaging \& Distributing};
        }
        \onslide<4->{
            \node (Z) [draw=mLightGreen, fit= (G) (H) (I) (J), inner sep=0.55cm, thick, fill=mLightGreen, fill opacity=0.1] {};
            \node [yshift=1.5ex, mLightGreen] at (Z.south) {Continuous Integration};
        }
    \end{tikzpicture}
    }
    \note {
        El desarrollo de una nueva funcionalidad se podría dividir en tres módulos:
        \begin{enumerate}
            \item Creación.
            \item Empaquetado y distribución.
            \item Integración continua.
        \end{enumerate}
    }
\end{frame}

\subsection{Simplified workflow}
\begin{frame}[fragile]{Simplified workflow I}
    \makebox[\textwidth][c]{
    \begin{tikzpicture}[
            node distance=0.8cm,
            >={Triangle[angle=60:1pt 2]},
            shorten >= 2pt,
            shorten <= 2pt,
            arrow/.style={
                ->,
                mLightBrown,
                line width=3pt
            }
        ]
        \ImageNode[label={Cookiecutter}]{A}{python_logo.png}
        \ImageNode[label={0:Code},right=of A]{B}{python_logo.png}
        \ImageNode[label={Clinner Build},above right=of B]{C}{python_logo.png}

        \ImageNode[label={-90:GitHub},below right=of B]{G}{github_logo.png}
        \ImageNode[label={-90:Travis},right=of G]{H}{travis_logo.png}
        \ImageNode[label={-90:Codecov},right=of H]{I}{codecov_logo.png}
        \ImageNode[label={0:ReadTheDocs},below=of H]{J}{readthedocs_logo.png}

        \draw[arrow] (A) -- (B);
        \draw[arrow] (B) -- (C);
        \draw[arrow] (B) -- (G);
        \draw[arrow] (G) -- (H);
        \draw[arrow] (H) -- (I);
        \draw[arrow] (G) -- (J);
    \end{tikzpicture}
    }
    \note{
        Los módulos anteriores son sustituidos por:
        \begin{enumerate}
            \item Creación, reemplazado por cookiecutter.
            \item Empaquetado y distribución, reemplazado por clinner.
            \item Integración continua, automático.
        \end{enumerate}
    }
\end{frame}

\begin{frame}[fragile]{Simplified workflow II}
    \begin{block}{Workflow execution}
        \begin{minted}{bash}
cookiecutter <project_name>
...code...
python build.py dist (patch|minor|major)
git push
        \end{minted}
        \note{
            \begin{enumerate}
                \item Crea una vez el proyecto con cookicutter.
                \item Desarrolla una nueva funcionalidad.
                \item Sube de versión, empaqueta y distribuye con un comando.
                \item Pushea tu código.
            \end{enumerate}

            Se dedica mucho más tiempo a desarrollar ya que todo el proceso de empaquetado y distribución es casi automático.
        }
    \end{block}
\end{frame}




\begin{frame}[standout]
    \Huge{Open source your code !}

    \Huge{\faLinux}
\end{frame}

\end{document}
