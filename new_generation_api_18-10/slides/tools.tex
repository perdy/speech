% Benefits
\section{Benefits}
\begin{frame}{Docs in sync with code}
    \textbf{\color{Teal}{GET}} \texttt{/puppy/}\\
    \textit{List all puppies !}\\
    Query params: \textbf{name}\\
    Response body: \textbf{List[PuppyType]}

    \vspace{3em}

    \textbf{\color{Teal}{POST}} \texttt{/puppy/}\\
    \textit{Register a new puppy !}\\
    Request body: \textbf{PuppyType}\\
    Response body: \textbf{PuppyType}
\end{frame}

\begin{frame}[fragile]{Mock the API}
    Views defined plain input parameters and output schema so that can be completely mocked.

    \begin{minted}[fontsize=\footnotesize]{python}
def list_puppy(name: PuppyName) -> typing.List[PuppyType]:
    """
    List all puppies !
    """
    pass
    \end{minted}
\end{frame}

\begin{frame}{Generate API Schema}
    All these changes made that API to expose all the information needed to automatically generate the schema.
    
    \textbf{Types} has a direct relation to \textbf{JSON Schema}, so each one can generate his own schema.

    The whole \textbf{API can be inspected} to build the schema based on standards like \textbf{OpenAPI} (former Swagger).
\end{frame}

\begin{frame}{Generic clients}
    Based on standard schemas such as \textbf{OpenAPI} it's quite easy to create generic clients for our API.
\end{frame}
