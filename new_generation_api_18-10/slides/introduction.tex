% Introduction 
\section{Introduction}
\begin{frame}{Introduction}
    What's wrong with current API frameworks? Nothing at all, except they seems a bit \textbf{old} and \textbf{unexpressive}.

    Let's improve it using new Python functionalities like:
    
    \begin{itemize}
        \item Type annotation.
        \item Module \texttt{typing}. 
        \item New \texttt{async}/\texttt{await} semantic.
        \item Module \texttt{asyncio}.
    \end{itemize}
\end{frame}

\begin{frame}{Introduction}
    The codebase can be very \textbf{expressive} in terms of describing the API. Make it the first information source of the API.
\end{frame}

\begin{frame}{Goals}
    \textbf{Speep up} API building and maintenance.

    Create an API with a codebase \textbf{expressive}.

    An \textbf{interactive documentation} kept \textbf{in sync} with the API.

    A way to infer and generate API \textbf{schema}.

    Possibility to work with \textbf{ASGI} and \textbf{websockets}.
\end{frame}

\begin{frame}{Example API}
    Puppy API \faPaw

    \begin{itemize}
        \item \textbf{Register} a new puppy.
        \item \textbf{List} all puppies, filtered by name.
    \end{itemize}
\end{frame}

\begin{frame}[fragile]{Example API}
    \begin{minted}[fontsize=\footnotesize]{python}
def puppy(request):
    if request.method == "POST":
        data = JSONParser().parse(request)
        serializer = PuppySerializer(data=data)
        if serializer.is_valid():
            serializer.save()
            return JsonResponse(serializer.data, status=201)
        return JsonResponse(serializer.errors, status=400)

    if request.method == "GET":
        puppies = Puppy.objects.all()
        name = request.query_params.get('name', None)
        if name is not None:
            puppies = puppies.filter(name=name)
        serializer = PuppySerializer(puppies, many=True)
        return JsonResponse(serializer.data, safe=False)
    \end{minted}
\end{frame}
