% Introduction 
\section{Introduction}
\subsection{Why should I distribute my application?}
\begin{frame}{Why should I distribute my application?}
    \begin{itemize}[<+- | alert@+>]
        \item Given enough eyeballs, all bugs are shallow.
        \item Upstream improvements.
        \item Force multiplier.
        \item Modular.
        \item Great advertising.
        \item Attract talent.
        \item Stand on the shoulders of giants.
        \item Best technical interview possible.
        \item Show your code.
    \end{itemize}
    \note{
        \begin{description}
            \item[Given enough eyeballs, all bugs are shallow]
            \item[Upstream improvements:] If you consume open source software, it's in your best interest to contribute back.
            \item[Force multiplier:] Diversity of ideas from exposing the problem.
            \item[Modular:] Open source projects tend to be more modularly architected.
            \item[Great advertising:] Maintainers of successful open source projects are often seen as industry leaders
            \item[Attract talent:] Developers want to work on yet-unsolved problems.
            \item[Stand on the shoulders of giants]
            \item[Best technical interview possible:] You can hire much more confidently if, for the past six months, the candidate has been contributing to the project you want them work on, and you like their work.
            \item[Show your code]
        \end{description}
    }
    \footnotetext[1]{\href{https://opensource.com/life/15/12/why-open-source}{https://opensource.com/life/15/12/why-open-source}}
\end{frame}

\subsection{Versioning}
\begin{frame}{Versioning}
    \begin{block}{Version schema}
        A normal version must be denoted by \texttt{X.Y.Z} where \texttt{X}, \texttt{Y} and \texttt{Z} are positive integers. \texttt{X} represents the major version, \texttt{Y} the minor version and \texttt{Z} the patch version. Version \texttt{1.0.0} defines the public API.
    \end{block}
    \pause
    \begin{block}{Version upgrade}
        Given a version number, increment:
        \begin{description}
            \item[Major] version when you make incompatible API changes. Reset minor and patch version to \texttt{0}.
            \item[Minor] version when you add functionality in a backwards-compatible manner. Reset patch version to \texttt{0}.
            \item[Patch] version when you make backwards-compatible bug fixes.
        \end{description}
    \end{block}
    \footnotetext[1]{\href{http://semver.org/}{http://semver.org/}}
\end{frame}

\subsection{Tools and Services}
\begin{frame}{Tools I}
    \begin{block}{Prospector}
        Static code analysis using different tools.
        \href{https://github.com/landscapeio/prospector}{https://github.com/landscapeio/prospector}
    \end{block}
    \pause
    \begin{block}{Sphinx}
        Create documentation for your project.
        \href{https://github.com/sphinx-doc/sphinx}{https://github.com/sphinx-doc/sphinx}
    \end{block}
    \pause
    \begin{block}{Bumpversion}
        Utility to upgrade your project version.
        \href{https://github.com/peritus/bumpversion}{https://github.com/peritus/bumpversion}
    \end{block}
    \pause
    \begin{block}{Pre-commit}
        Utility that does some checks before git commits.
        \href{https://github.com/pre-commit/pre-commit}{https://github.com/pre-commit/pre-commit}
    \end{block}
\end{frame}
\begin{frame}{Tools II}
    \begin{block}{Tox}
        Run your tests using many different python interpreters.
        \href{https://github.com/tox-dev/tox}{https://github.com/tox-dev/tox}
    \end{block}
    \pause
    \begin{block}{Clinner}
        Utility to create powerful Command Line Interfaces with a few lines.
        \href{https://github.com/PeRDy/clinner}{https://github.com/PeRDy/clinner}
    \end{block}
    \pause
    \begin{block}{Cookiecutter}
        Application that creates project skeletons using Jinja templates.
        \href{https://github.com/audreyr/cookiecutter}{https://github.com/audreyr/cookiecutter}
    \end{block}
    \pause
    \begin{block}{Cookiecutter Template}
        Cookiecutter template for Python packages.
        \href{https://github.com/PeRDy/cookiecutter-python-package}{https://github.com/PeRDy/cookiecutter-python-package}
    \end{block}
    \note{
        Cookiecutter reduces the task of creates the whole project structure. 

        Clinner simplifies the process of create a package and upload it.
        
        Tox will ease the testing. 

        Clinner is the cornerstone around the rest of tools, the build script done as example in the doc allows to define a single entrypoint for the rest of tools.
        \href{http://clinner.readthedocs.io/en/latest/examples.html#builder-main}{http://clinner.readthedocs.io/en/latest/examples.html#builder-main}.

        A deeper explanation about the cookiecutter template and the project hierarchy will come in next sections.

        \emph{Distutils}: setup.py uses distutils.
    }
\end{frame}

\begin{frame}{Services}
    \begin{block}{PyPI}
        Python Package Index, the main repository of python software.
        \href{https://pypi.python.org/pypi}{https://pypi.python.org/pypi}
    \end{block}
    \pause
    \begin{block}{GitHub}
        Repositories for your source code.
        \href{https://github.com}{https://github.com}
    \end{block}
    \pause
    \begin{block}{Travis}
        Continuous Integration service.
        \href{https://travis-ci.org/}{https://travis-ci.org/}
    \end{block}
    \pause
    \begin{block}{Coveralls}
        Keeps the changes of test coverage of your code.
        \href{https://coveralls.io}{https://coveralls.io}
    \end{block}
    \pause
    \begin{block}{ReadTheDocs}
        Stores and serves documentation for your project.
        \href{https://readthedocs.io}{https://readthedocs.io}
    \end{block}
    \note{
        Alternatives to GitHub: \emph{Bitbucket}, \emph{Gitlab}.

        \emph{Travis} and \emph{Coveralls} are free for your open source projects.
    }
\end{frame}


