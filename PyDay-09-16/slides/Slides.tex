\documentclass[final, 9pt, svgnames]{beamerPerdy}
\usetheme[
  bullet=square,			% Modelo de los items de listados (circle, square)
  pagenumber,				% Numero de paginas en la barra inferior
  headline=true,			% Barra de navegacion superior (sect y subsect)
  watermark=img/data-analysis.png,		% Imagen para el fondo
  background=Amarillo         % Color del degradado del fondo
]{Perdy}

\setdefaultlanguage{english}

% Comando para usar comentarios
\newcommand{\comment}[1]{\textcolor{comment}{\footnotesize{#1}\normalsize}}


\title{Python Data Analysis and Plotting}
%\subtitle{Sistemas Neurodifusos}
\author[J. A. Perdiguero López]{José Antonio Perdiguero López}
\date{\today}
\institute[Málaga Python]{Málaga Python: PyDay 2016}

\begin{document}

\setcounter{tocdepth}{2}

% Titulo
\begin{frame}
  \titlepage
\end{frame}

% Indice
\begin{frame}
  \frametitle{Index}
  \tableofcontents
\end{frame}

% Documento
\section{Introduction}
  \subsection{What is Data Analysis?}
  \begin{frame}{What is Data Analysis?}
    \begin{itemize}[<+->]
      \item Planning the \emph{gathering of data} to make its analysis easier, more precise or more accurate.
      \item Procedures for \emph{analyzing data}.
      \item Techniques for \emph{interpreting the results} of such procedures.
      \item All the machinery and results of \emph{mathematical statistics} which apply to analyzing data.
    \end{itemize}
  \end{frame}

  \subsection{Why Python?}
  \begin{frame}{Why Python?}
    \begin{itemize}[<+->]
      \item Python as Glue.
      \item Solves the "two-languages" problem: Prototype vs Production.
      \item Increase programming effectiveness (at a CPU time cost?).
    \end{itemize}
  \end{frame}
  \subsection{Python Data Analysis Stack}
  \begin{frame}{Python Data Analysis Stack}
    \begin{itemize}[<+->]
      \item NumPy (\href{https://www.numpy.org}{https://www.numpy.org})
      \item Pandas (\href{https://pandas.pydata.org}{https://pandas.pydata.org})
      \item Plotly (\href{https://plot.ly}{https://plot.ly})
    \end{itemize}
  \end{frame}
\section{Pandas}
  \subsection{Data Structures}
  \begin{frame}{Data Structures}
    \begin{block}{Series}
      One-dimensional labeled array capable of holding any data type.
    \end{block}

    \begin{block}{DataFrame}
      Two-dimensional labeled data structure with columns of potentially different types.
    \end{block}
  \end{frame}

  \subsection{Input/Output}
  \begin{frame}{Input}
    Pandas DataFrame accepts many different kinds of input:
		\begin{itemize}
			\item Dict of 1D ndarrays, lists, dicts, or Series.
			\item 2-D numpy.ndarray.
			\item Structured or record ndarray.
			\item A Series.
			\item Another DataFrame.
		\end{itemize}
  \end{frame}
  \begin{frame}{Read}
		DataFrames can be loaded from different sources:
		\begin{itemize}
			\item csv
			\item excel
			\item hdf
			\item sql
			\item json
			\item msgpack (experimental)
			\item html
			\item gbq (experimental)
			\item stata
			\item sas
			\item clipboard
			\item pickle
		\end{itemize}
  \end{frame}
  \begin{frame}{Write}
		DataFrames can be directly writted to:
		\begin{itemize}
			\item csv
			\item excel
			\item hdf
			\item sql
			\item json
			\item msgpack (experimental)
			\item html
			\item gbq (experimental)
			\item stata
			\item clipboard
			\item pickle
		\end{itemize}
  \end{frame}

  \defverbatim[colored]\selectCode{
		\begin{codigo}
		\begin{pythoncode}
# Get rows from 5 to 10
df.ix[5:10]

# Get column Foo
df.ix[:, 'Foo']

# Get rows 1, 3 and 7; and columns Foo and Bar
df.ix[[1, 3, 7], ['Foo', 'Bar']]
		\end{pythoncode}
		\end{codigo}
	}

	\defverbatim[colored]\filterCode{
		\begin{codigo}
		\begin{pythoncode}
# Get rows whose value of Foo column is positive
df[df['Foo'] > 0]
		\end{pythoncode}
		\end{codigo}
	}

	\defverbatim[colored]\selectAndFilterCode{
		\begin{codigo}
		\begin{pythoncode}
# Get Bar column for those rows whose value of Foo column is positive
df.ix[df['Foo'] > 0, 'Bar']
		\end{pythoncode}
		\end{codigo}
	}

  \subsection{Select and Filter}
  \begin{frame}{Select and Filter}
		\begin{block}{Select}
			Select rows, columns or cells using python indexing notation:
			\selectCode
		\end{block}
		\begin{block}{Filter}
			Apply filters to DataFrames using python expressions:
			\filterCode
		\end{block}
		\begin{block}{Select and Filter}
			Select rows using filters and index:
			\selectAndFilterCode
		\end{block}
  \end{frame}

  \subsection{Merge}
  \begin{frame}{Merge Structures}
		There are two ways of merge pandas DataFrame structures:
		\begin{description}
			\item[Concat:] Add a DataFrame just at the end.
			\item[Merge:] SQL-style merges using join.
		\end{description}
  \end{frame}

  \subsection{Group By}
  \begin{frame}{Group By}
		The process of \emph{Group By} involves the steps:
		\begin{description}
			\item[Split] the data into groups based on some criteria.
			\item[Apply] a function to each group independently.
			\item[Combine] the results into a data structure.
		\end{description}
  \end{frame}
\end{document}
